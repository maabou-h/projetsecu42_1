%******************************************************************************%
%                                                                              %
%                  sample.en.tex for LaTeX                                     %
%                  Created on : Tue Mar 10 13:27:28 2015                       %
%                  Made by : David "Thor" GIRON <thor@42.fr>                   %
%                  Updated by : Catherine "Cat" MADINIER <cat@42.fr>           %
%                                                                              %
%******************************************************************************%

\documentclass{42-en}


%******************************************************************************%
%                                                                              %
%                                    Header                                    %
%                                                                              %
%******************************************************************************%
\begin{document}

                           \title{ft\_malcolm}
                          \subtitle{An introduction to Man in the Middle attacks}
\summary{This is the first project of a network security branch created by maabou-h.}

\maketitle

\tableofcontents


%******************************************************************************%
%                                                                              %
%                                  Foreword                                    %
%                                                                              %
%******************************************************************************%
\chapter{Foreword}

    Yes, no, maybe\\
    I don't know\\
    Can you repeat the question?\\
    You're not the boss of me now\\
    You're not the boss of me now\\
    You're not the boss of me now, and you're not so big\\
    You're not the boss of me now\\
    You're not the boss of me now\\
    You're not the boss of me now, and you're not so big\\
    You're not the boss of me now, and you're not so big
    Life is unfair, so i just stare at the stain on the wall where\\
    The tv'd been, but ever since we've moved in it's been empty\\
    Why i, why i'm in this room\\
    There is no point explaining\\
    You're not the boss of me now, and you're not so big\\
    You're not the boss of me now\\
    You're not the boss of me now\\
    You're not the boss of me now, and you're not so big\\
    Malcolm in the middle, and i confess\\
    I like this mess i've made so far\\
    Grade on a curve and you'll observe\\
    I'm right below the horizon\\
    Yes,… \\

    This subject has nothing to do with Malcolm in the middle, but rather with Man in the middle!\\
    \newpage
    
%******************************************************************************%
%                                                                              %
%                                 Introduction                                 %
%                                                                              %
%******************************************************************************%
\chapter{Introduction}

    In this first project of network security, 
     you will implement the Address Resolution Protocol spoofing/poisoning method, which is one of the most basic Man In The Middle attacks 
     using a vulnerability present in the ARP protocol.



%******************************************************************************%
%                                                                              %
%                                  Goals                                       %
%                                                                              %
%******************************************************************************%
\chapter{Goals}

    In this first project aimed at introducing you to network security,
     you will discover in detail what is the Address Resolution Protocol,
      that you may have seen during your days of piscine as ARP, and how it 
      works. You will find that this implementation has several vulnerabilities 
      and while protections and alternatives to arp might exist,
       it remains widely used and is unsafe and unprotected in most cases.

\info{
              You should really start by reading the RFC 826 and 7042
            }

%******************************************************************************%
%                                                                              %
%                             General instructions                             %
%                                                                              %
%******************************************************************************%
\chapter{General instructions}

    • This project will be corrected by humans only. You’re allowed to organise and name your files as you see fit, but you must follow the following rules\\
    \\
    • You must use C and submit a Makefile\\
    \\
    • You are allowed to use one global variable within your project.\\
    \\
    • Your Makefile must compile the project and must contain the usual rules. It must recompile and re-link the program only if necessary.\\
    \\
    • You have to handle errors carefully. In no way can your program quit in an unexpected manner (Segmentation fault, bus error, double free, etc).\\
    \\
    • Within the mandatory part, you are allowed to use the following functions:\\

          ◦ sendto, recvfrom\\

          ◦ socket, setsockopt\\  

          ◦ getuid, signal, close\\

          ◦ inet\_addr, gethostbyname, getifaddrs, freeifaddrs\\ 

          ◦ htons, ntohs\\  

          ◦ printf and its family.\\  

          ◦ Your libft functions.\\    

          ◦ You are allowed to use other functions to complete the bonus part as long as their use is justified during your defense. Be smart.\\




%******************************************************************************%
%                                                                              %
%                             Mandatory part                                   %
%                                                                              %
%******************************************************************************%
\chapter{Mandatory part}

    Your program must take the following four arguments:\\
    source ip\\
    source mac address\\
    target ip\\
    target mac address\\
    Those parameters must always be given in that order.\\

        Example usage:
               \begin{42console}
    $./ft_malcolm 10.11.254.254 aa:bb:cc:dd:ee:ff 10.11.12.18 42:42:42:42:42:42 \end{42console}

    You must only manage a simple IPv4 address for the mandatory part.\\
    If the arguments provided do not follow that rule or are invalid, your program should exit and output some information.\\
    How precise the information is will be up to you, but you are encouraged to provide details on the errors.\\

    For this project, you will have to send an ARP reply containing the source ip and the mac you want to tell it belongs to
    to spoof the arp table of the target.\\
    When run, your program will have to wait for an ARP request before sending a single ARP reply and exit.\\
    If everything went well, the arp table on the target should contain the associated ip and mac you provided as source.
    \warn {
              You are only allowed to spoof IPs that belong to you, such as your VM. Spoofing other IPs might result in problems and/or sanctions.
            }
    Your program should be able to exit when the user inputs Ctrl+C


%******************************************************************************%
%                                                                              %
%                                 Bonus part                                   %
%                                                                              %
%******************************************************************************%
\chapter{Bonus part}
 \warn {
    We will look at your bonuses if and only if your mandatory part isEXCELLENT. This means that your must complete the mandatory part,beginning to end, and your error management must be flawless, even incases of twisted or bad usage.  If that’s not the case, your bonuseswill be totally IGNORED.
 }

 Find below a few ideas of interesting bonuses:\\
 
    • IPv6 management.
    • Decimal notation for IPv4 addresses
    • Verbose mode to print packet information
    • Any other bonuses that provide extra functionalities to your program


%******************************************************************************%
%                                                                              %
%                           Turn-in and peer-evaluation                        %
%                                                                              %
%******************************************************************************%
\chapter{Turn-in and peer-evaluation}

        • Submit your work on your GiT repository as usual. Only the work on your repository will be graded.\\
        \newline
        • You have to be in a VM with a Linux kernel > 3.14. Note that grading was designed on a Debian 7.0 stable.



%******************************************************************************%
\end{document}
