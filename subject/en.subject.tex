%******************************************************************************%
%                                                                              %
%                  sample.en.tex for LaTeX                                     %
%                  Created on : Tue Mar 10 13:27:28 2015                       %
%                  Made by : David "Thor" GIRON <thor@42.fr>                   %
%                  Updated by : Catherine "Cat" MADINIER <cat@42.fr>           %
%                                                                              %
%******************************************************************************%

\documentclass{42-en}


%******************************************************************************%
%                                                                              %
%                                    Header                                    %
%                                                                              %
%******************************************************************************%
\begin{document}



                           \title{Sample document}
                          \subtitle{Awesome subtitle}
\summary {
  This document is a sample and an introduction to \texttt{LaTeX} and
  the style class homemade by \href{www.42.fr}{42}.
}

\maketitle

\tableofcontents


%******************************************************************************%
%                                                                              %
%                                  Foreword                                    %
%                                                                              %
%******************************************************************************%
\chapter{Foreword}

    The foreword section of a \texttt{42} subject is usually not
    related in any way to the actual topic of the subject. The idea is
    to share some jokes (often questionable) or something that the
    community might be interested in.\\





    % Spacing in the source code does not influence spacing in the
    % generated pdf. The blank lines aboves and below won't appear.
    % Instead, use \newline (or its shortcut \\) and \newpage to
    % create vertical spacing.





    So, let's use the foreword section of this \texttt{42} subject
    template to introduce the content of this document and
    its goals. Specifically, the formatting of a trivial
    \texttt{LaTeX} document and the normalized chaptering of our
    subjects. If you read this from the pdf, don't forget to open the
    source file (\texttt{sample.en.tex} file) next to it, in
    order to see behind the scenes and to understand what command
    generates what result. Otherwise, if you have started with the
    sources, congrats, that's the spirit! But open the pdf
    (\texttt{sample.en.pdf} file) anyway to double check.\\

    What can you do if the \texttt{sample.en.pdf} file is not available?
    Easy, just compile the source file \texttt{sample.en.tex} using
    the shell command \texttt{make}. Please refer to the documentation
    to set up \texttt{LaTeX} on your system if needed.\\

    If you're not familiar with \texttt{LaTeX}'s syntax, here is a
    fairly comprehensive list of everything you'll need to write your
    subject.\\


    \section{Example of section}


        \subsection{Example of sub-section}

           This sub-section is empty.


        \newpage


        \subsection{A bullet point list}

            \begin{itemize}\itemsep1pt
                \item what
                \item a
                \item wonderful
                \item list.\\
            \end{itemize}


        \subsection{A descriptions list}

            \begin{description}\itemsep3pt
                \item [Orange:] Round and orange fruit.
                \item [Strawberry:] Strawberry shaped fruit. Also red.
                \item [Cucumber:] Phallus-shaped green vegetable.\\
            \end{description}


        \subsection{An enumeration}

            An enumeration of the reasons why I like you:\\

            \begin{enumerate}\itemsep7pt
                \item You are smart.
                \item Your are very talented.
                \item Your are magnificent.
                \item I'm a nice person.
            \end{enumerate}


        \subsection{Urls and links}

            If you have no clue how to insert links or urls in your
            document, search for an online explanation using
            \href{www.google.com}{Google}. Please note that \texttt{Google}
            is available at \url{www.google.com}.


        \newpage


        \subsection{An info box}

            \info{
              For information, please read this info box.
            }


        \subsection{A hint box}

            \hint {
              You should read this hint box, really.
            }


        \subsection{A warning box}

            \warn {
              Beware! This is a warning box!
            }


        \newpage


        \subsection{A \texttt{shell} snippet}

           \begin{42console}
$sudo rm -rf /\end{42console}



        \subsection{A \texttt{C} code snippet}

           \begin{42ccode}
int main( void ) {

    puts( "hello world !" );
    return 0;
}
\end{42ccode}


        \subsection{A \texttt{C++} code snippet}

            \begin{42cppcode}
int main( void ) {

    std::cout << "hello world !" << std::endl;
    return 0;
}
\end{42cppcode}


        \subsection{A picture !}

            \begin{figure}[H]
                \begin{center}
                \end{center}
            \end{figure}


        \newpage


        \subsection{Some special characters}

            \begin{description}\itemsep1pt
                \item [Underscore:] \_
                \item [Ampersand:] \&
                \item [Dollar:] \$
                \item [Elipsis:] \dots
            \end{description}


    \section{About chaptering}

    Each chapter of the pdf must appear in your subject,
    \textbf{including} the \texttt{Foreword} chapter. For your
    convenience, the best way to use this template \texttt{LaTeX} file is
    to copy it and rename it, then replace the provided descriptions by
    your own content.\\

    \warn{
      If you are part of a company, the \texttt{Foreword} chapter is
      the best place to write about your business, the context
      of this project, introduce yourself and/or your team, etc.
    }

%******************************************************************************%
%                                                                              %
%                                 Introduction                                 %
%                                                                              %
%******************************************************************************%
\chapter{Introduction}

    The introduction is a presentation of the project outline. It's good
    practice to specify some context and some ideas about what needs to be
    done. So with these few lines, students get a global overview
    of the addressed topics.



%******************************************************************************%
%                                                                              %
%                                  Goals                                       %
%                                                                              %
%******************************************************************************%
\chapter{Goals}

    This chapter explains the educational interest of your project,
    because in the end a project is only a means to discover and
    explore new topics. Take our \texttt{42} \texttt{C++}
    project \texttt{Nibbler}. Despite looking like a simple
    \texttt{Snake} game, this project introduces the students to the
    creation of an API and additional plugins for a \texttt{C++} program.



%******************************************************************************%
%                                                                              %
%                             General instructions                             %
%                                                                              %
%******************************************************************************%
\chapter{General instructions}

    This chapter lists all the basic instructions of a project.
    Languages, restrictions, authorizations, compilation, etc.



%******************************************************************************%
%                                                                              %
%                             Mandatory part                                   %
%                                                                              %
%******************************************************************************%
\chapter{Mandatory part}

    Heart of the subject, the mandatory part describes in detail the
    work expected and the potentially required tools and/or technologies.
    The secret of a good subject is the balance between
    being specific and leaving some room for interpretation and
    imagination. This balance is very important as it is what will
    fuel debates and argumentations during peer-evaluations.



%******************************************************************************%
%                                                                              %
%                                 Bonus part                                   %
%                                                                              %
%******************************************************************************%
\chapter{Bonus part}

    When students have invested time and efforts in a project and have reached
    their goals, it's natural to want to go further! The bonus section
    is here to nurture these ambitions. Of course, the bonus part is
    available if and only if the mandatory part is complete and flawless.



%******************************************************************************%
%                                                                              %
%                           Turn-in and peer-evaluation                        %
%                                                                              %
%******************************************************************************%
\chapter{Turn-in and peer-evaluation}

    This part describes the conditions and instructions for the turn-in and
    peer-evaluation of the project. If your project does not
    require any specific turn-in or peer-evaluation instruction, feel free to
    use the following paragraph as is:\\

    Turn in your work using your \texttt{GiT} repository, as
    usual. Only the work that's in your repository will be graded during
    the evaluation.



%******************************************************************************%
\end{document}
