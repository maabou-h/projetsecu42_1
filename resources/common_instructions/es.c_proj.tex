%******************************************************************************%
%                                                                              %
%                        Common Instructions                                   %
%                          for C Projects                                      %
%                                                                              %
%******************************************************************************%

\chapter{Reglas comunes}
    \begin{itemize}
      \item Su proyecto debe estar programado respetando la Norma. Si tiene archivos o
        funciones extras, entrar�n dentro de la verificaci�n de la norma y, como haya 
		alg�n error de norma, tendr� un 0 en el proyecto.

      \item Sus funciones no pueden pararse de forma inesperada (segmentation
      fault, bus error, double free, etc.) salvo en el caso de un comportamiento
      indefinido. Si esto ocurre, se considerar� que su proyecto no es funcional y
	  tendr� un 0 en el proyecto.

      \item Cualquier memoria reservada en el mont�n (heap) tendr� que ser liberada cuando sea
	  necesario. No se tolerar� ninguna fuga de memoria.

      \item Si el proyecto lo requiere, tendr� que entregar un Makefile que compilar� sus 
        c�digos fuente para crear la salida solicitada, utilizando los flags \texttt{-Wall},
        \texttt{-Wextra} y \texttt{-Werror}. Su Makefile no debe hacer relink.

      \item Si el proyecto requiere un Makefile, su Makefile debe incluir 
        al menos las reglas \texttt{\$(NAME)}, \texttt{all}, \texttt{clean},
        \texttt{fclean} y \texttt{re}.

      \item Para entregar los extras, debe incluir en su Makefile una regla \texttt{bonus} que
        a�adir� los headers, bibliotecas o funciones que no est�n permitidos en la parte principal
		del proyecto. Los extras deben estar dentro de un archivo \texttt{\*\_bonus.\{c/h\}}.
		Las evaluaciones de la parte obligatoria y de la parte extra se hacen por separado.

      \item Si el proyecto autoriza su \texttt{libft}, debe copiar sus c�digos fuente y 
        y su Makefile asociado en un directorio libft, dentro de la ra�z.
        El Makefile de su proyecto debe compilar la biblioteca con la ayuda de su Makefile
        y despu�s compilar el proyecto.

      \item Le recomendamos que cree programas de prueba para su proyecto, aunque ese
        trabajo \textbf{no ser� ni entregado ni evaluado}. Esto le dar� la oportunidad
        de probar f�cilmente su trabajo al igual que el de sus compa�eros. 

      \item Deber� entregar su trabajo en el git que se le ha asignado. Solo se evaluar�
        el trabajo que se suba al git. Si Deepthought debe corregir su trabajo, lo har�
        al final de las evaluaciones por sus pares.
        Si surge un error durante la evaluaci�n Deepthought, esta �ltima se parar�.
    \end{itemize}
