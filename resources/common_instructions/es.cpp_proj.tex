%******************************************************************************%
%                                                                              %
%                  instructions.tex for 42's Piscine C++                       %
%                  Created on : Mon Sep  8 15:57:19 2014                       %
%                  Made by : David "Thor" GIRON <thor@42.fr>                   %
%                                                                              %
%******************************************************************************%


\chapter{Reglas Generales}


    \begin{itemize}
        \item La declaraci�n de una funci�n en un header (excepto para los templates)
            o la inclusi�n de un header no protegido conllevar� un 0 en el ejercicio.
        
		\item Salvo que se indique lo contrario, cualquier salida se mostrar� en stdout y terminar� con un newline.
		\item Los nombres de ficheros impuestos deben seguirse escrupulosamente, as� como los nombres de clase, de funci�n y de m�todo.
		\item Recordatorio : ahora est� codificando en \texttt{C++}, no en
            \texttt{C}. Por eso :
            \begin{itemize}
                \item Las funciones siguientes est�n \textbf{PROHIBIDAS}, y su uso conllevar� un 0: \texttt{*alloc}, 
                    \texttt{*printf} et \texttt{free}
                \item Puede utilizar pr�cticamente toda la librer�a est�ndar. NO OBSTANTE, ser�a m�s inteligente intentar usar la versi�n para C++ que a lo que ya est� acostumbrado en C, para no basarse en lo que ya ha asimilado.
                    Y no est� autorizado a utilizar la STL hasta que le llegue el momento de trabajar con ella (m�dulo 08).
                    Eso significa que hasta entonces no se puede utilizar Vector/List/Map/etc... ni nada similar que requiera un include <algorithm>.
		  \end{itemize}
        \item El uso de una funci�n o de una mec�nica expl�citamente prohibida ser� sancionado con un \texttt{0}
        \item Tenga tambi�n en cuenta que, a menos que se autorice de manera expresa, las palabras clave
          \texttt{using namespace} y \texttt{friend} est�n prohibidas.
          Su uso ser� castigado con un \texttt{0}.
        \item Los ficheros asociados a una clase se llamar�n siempre \texttt{ClassName.cpp}
          y \texttt{ClassName.hpp}, a menos que se indique otra cosa.
        \item Tiene que leer los ejemplos en detalle. Pueden contener prerrequisitos no indicados en las instrucciones.
        \item No est� permitido el uso de librer�as externas, de las que forman parte \texttt{C++11}, \texttt{Boost}, ni ninguna de las herramientas que ese amigo suyo que es un figura le ha recomendado.
        \item Probablemente tenga que entregar muchos ficheros de clase, lo que le va a parecer repetitivo hasta que aprenda a hacer un script con su editor de c�digo favorito.
        \item Lea cada ejercicio en su totalidad antes de empezar a resolverlo.
        \item El compilador es \texttt{clang++}
        \item Se compilar� su c�digo con los flags \texttt{-Wall -Wextra -Werror}
        \item Se debe poder incluir cada include con independencia de los dem�s include. Por lo tanto, un include debe incluir todas sus dependencias.
        \item No est� obligado a respetar ninguna norma en \texttt{C++}. Puede utilizar el estilo que prefiera. Ahora bien, un c�digo ilegible es un c�digo que no se puede calificar.
        \item Importante: no va a ser calificado por un programa (a menos que el enunciado especifique lo contrario). Eso quiere decir que dispone cierto grado de libertad en el m�todo que elija para resolver sus ejercicios. 
        \item Tenga cuidado con las obligaciones, y no sea z�ngano; podr�a dejar escapar mucho de lo que los ejercicios le ofrecen.
        \item Si tiene ficheros adicionales, no es un problema. Puede decidir separar el c�digo de lo que se le pide en varios ficheros, siempre que no haya moulinette.
        \item Aun cuando un enunciado sea corto, merece la pena dedicarle algo de tiempo, para asegurarse de que comprende bien lo que se espera de usted, y de que lo ha hecho de la mejor manera posible.

\end{itemize}

%******************************************************************************%
\newpage
