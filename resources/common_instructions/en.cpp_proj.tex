%******************************************************************************%
%                                                                              %
%                  instructions.tex for 42's Piscine C++                       %
%                  Created on : Mon Sep  8 15:57:19 2014                       %
%                  Made by : David "Thor" GIRON <thor@42.fr>                   %
%                                                                              %
%******************************************************************************%


\chapter{General rules}


    \begin{itemize}

		\item Any function implemented in a header (except in the case of
		templates), and any unprotected header, means 0 to the exercise.

		\item Every output goes to the standard output, and will be
          ended by a newline, unless specified otherwise.

		\item The imposed filenames must be followed to the letter, as
          well as class names, function names and method names.

		\item Remember: You are coding in \texttt{C++} now, not in
          \texttt{C} anymore. Therefore:
		  
		  \begin{itemize}
		  
		  \item	The following functions are FORBIDDEN, and their use will be
		  punished by a \texttt{0}, no questions asked: \texttt{*alloc},
		  \texttt{*printf} and \texttt{free}.

		  \item You are allowed to use basically everything in the standard
		  library. HOWEVER, it would be smart to try and use the C++-ish
		  versions of the functions you are used to in C, instead of just
		  keeping to what you know, this is a new language after all. And NO,
		  you are not allowed to use the STL until you actually are supposed to
		  (that is, until module 08). That means no vectors/lists/maps/etc... or
		  anything that requires an include <algorithm> until then.

		  \end{itemize}

		\item Actually, the use of any explicitly forbidden function or
		mechanic will be punished by a \texttt{0}, no questions asked.

        \item Also note that unless otherwise stated, the \texttt{C++}
          keywords \texttt{"using namespace"} and \texttt{"friend"} are
          forbidden. Their use will be punished by a \texttt{-42}, no
          questions asked.

        \item Files associated with a class will always be
          \texttt{ClassName.hpp} and \texttt{ClassName.cpp}, unless
          specified otherwise.

        \item Turn-in directories are \texttt{ex00/}, \texttt{ex01/},
          \dots, \texttt{exn/}.

        \item You must read the examples thoroughly. They can contain
          requirements that are not obvious in the exercise's
          description. If something seems ambiguous, you don't
          understand \texttt{C++} enough.

        \item Since you are allowed to use the \texttt{C++} tools you
          learned about since the beginning, you are not
          allowed to use any external library. And before you ask,
          that also means no \texttt{C++11} and derivates, nor
          \texttt{Boost} or anything your awesomely skilled friend
          told you \texttt{C++} can't exist without.

        \item You may be required to turn in an important number of
          classes. This can seem tedious, unless you're able to script
          your favorite text editor.

        \item Read each exercise FULLY before starting it! Really, do it.

        \item The compiler to use is \texttt{clang++}.

        \item Your code has to be compiled with the following
          flags : \texttt{-Wall -Wextra -Werror}.

        \item Each of your includes must be able to be included
          independently from others. Includes must contains every
          other includes they are depending on, obviously.

        \item In case you're wondering, no coding style is enforced
          during in \texttt{C++}. You can use any style you
          like, no restrictions. But remember that a code your
          peer-evaluator can't read is a code she or he can't grade.

	    \item Important stuff now : You will NOT be graded by a program,
		unless explictly stated in the subject. Therefore, you are afforded
		a certain amount of freedom in how you choose to do the exercises.
		However, be mindful of the constraints of each exercise, and DO NOT
		be lazy, you would miss a LOT of what they have to offer !

		\item It's not a problem to have some extraneous files in what you turn
		in, you may choose to separate your code in more files than what's
		asked of you. Feel free, as long as the result is not graded by a program.
              
		\item Even if the subject of an exercise is short, it's worth spending
		some time on it to be absolutely sure you understand what's expected of
		you, and that you did it in the best possible way.

        \item By Odin, by Thor! Use your brain!!!

\end{itemize}

%******************************************************************************%
\newpage
